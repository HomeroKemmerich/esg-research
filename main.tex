\documentclass{article}
\bibliography{references}

\begin{document}
	\section*{Introdução}
	ESG é a sigla para Environmental, Social and Governance (Ambiental, Social e Governança), e se refere a um conjunto de práticase princípios que as empresas podem adotar para demonstrar seu compromisso com a sustentabilidade e a responsabilidade social. Seus pilares são caracterizados da seguinte forma:
	\begin{itemize}
		\item Ambiental: como a empresa se relaciona com o meio ambiente, abrangendo questões como utilização de recursos naturais, emissão de poluentes e impactos ambientais em geral.
		\item Social: como a empresa se relaciona com seus funcionários, clientes e a comunidade onde está inserida, e abrange questões como diversidade, inclusão e bem-estar dos envolvidos.
		\item Governança: se refere à forma como a empresa é administrada, e abrange questões como a transparência e ética corporativa, assim como a sua estruturação hierárquica.
	\end{itemize}
	Nos últimos anos, o ESG vem ganhando força devido ao aumento da conscientização sobre a sustentabilidade e da responsabilidade social das empresas, algo que impacta inclusive a decisão de investidores para uma melhor alocação de capital.
	
	É possível observar que empresas que adotam o ESG possuem uma boa reputação perante o público, e tais iniciativas podem ser um indicativo da saúde corporativa da empresa, já que exigem planejamento e execução bem estruturados para serem realizados.
	
	Embora não existam legislações sobre o ESG, as normas preexistentes podem guiar a adoção dessas práticas para as empresas, tendo em vista que tangem assuntos semelhantes. Um exemplo disso é o Sistema de Gestão Ambiental da ISO 14000, um conjunto de regras que estabelece, um framework para que as empresas gerenciem seu impacto ambiental, algo de extrema importância no pilar ambiental do ESG.
	
	Em paralelo a isso, a ONU estabeleceu 17 objetivos para serem alcançados pelos países até 2030 (Objetivos de Desenvolvimento Sustentável – ODS). Muitos destes objetivos estão em alinhamento com o ESG, pois buscam o progresso de forma sustentável e socialmente justa.
	
	O ESG, por ser um conceito bastante abrangente, pode ser implementado em qualquer setor da economia, o que não deixa de fora as atividades econômicas mais recentes. Portanto, empresas de tecnologia, que já possuem inovação e questões sociais atuais em pauta, são alguns dos setores com mais facilidade de adotar tais práticas.
	Estudo de caso: empresas de tecnologia e provedores de serviços de nuvem
	
	Considerando um crescimento de 15\%, metade do ritmo dos últimos cinco anos, os ativos de ESG sob gestão poderiam subir para mais de um terço do total global projetado de US\$ 140,5 trilhões até 2025. Os ativos de ESG estão no caminho para alcançar US\$ 53 trilhões, com base em nossa análise, um aumento em relação aos US\$ 37,8 trilhões registrados até o final do ano. Eles saltaram de US\$ 30,6 trilhões em 2018 para US\$ 22,8 trilhões em 2016. \cite{}
	
	Uma pesquisa recente da Harris realizada em nome do Google Cloud indica que apenas 22\% das organizações têm um programa oficial de ESG em vigor e estão mensurando seu impacto, uma queda de 3\% em relação ao ano anterior. Isso coloca os executivos em uma posição desconfortável quando se trata de suas declarações públicas, com 87\% dos entrevistados afirmando que desejam que suas organizações tenham melhores ferramentas para medir os esforços de sustentabilidade, a fim de estabelecer metas mais precisas.
\end{document}

